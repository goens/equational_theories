\chapter{The Asterix equation}\label{asterix-chapter}

A \emph{translation-invariant magma} is a magma whose carrier $G$ is an abelian group $G = (G,+)$, and whose magma operation takes the form
$$ y \op x = x + f(x-y)$$
for some function $f: G \to G$.  Thus the translations on $G$ become magma isomorphisms.

For translation-invariant magmas, an equational law simplifies to a univariate functional equation.  For instance, writing $x = y+h$, we have
$$ y \op x = x + f(h)$$
$$ x \op (y \op x) = x + f(h) + f^2(h)$$
$$ y \op (x \op (y \op x)) = x + f(h) + f^2(h) + f( h + f(h) + f^2(h) )$$
where $f^2 = f \circ f$, so the Asterix equation (\Cref{eq65}) for such magmas simplifies to the univariant functional equation
\begin{equation}\label{fh}
   f(h) + f^2(h) + f( h + f(h) + f^2(h) ) = 0
\end{equation}
for $h \in G$.

This equation has some degenerate solutions, for instance we can take $f(h) = c$ for any constant $c$ of order $3$ in $G$.  It is challenging to construct more interesting solutions to this equation; however, we can do this if $G=\Z$ by a greedy algorithm.  We need the following technical definition.

\begin{definition}\label{partial-solution}  A \emph{partial solution} $(E_0, E_1, E_2, f)$ to \eqref{fh} consists of nested finite sets
$$ E_0 \subset E_1 \subset E_2 \subset \Z$$ together with a function $f: E_1 \to E_2$ with the following properties:
\begin{itemize}
  \item[(a)] If $h \in E_0$, then $f(h) \in E_1$, so that $f^2(h)$ is well-defined as an element of $E_2$; furthermore, $h + f(h) + f^2(h)$ lies in $E_1$, so that the left-hand side of \eqref{fh} makes sense; and \eqref{fh} holds.
  \item[(b)] The function $f$ is a bijection from $E_1 \backslash E_0$ to $E_2 \backslash E_1$.
\end{itemize}

We partially order the space of partial solutions to \eqref{fh} by writing $(E_0, E_1, E_2, f) \leq (E'_0, E'_1, E'_2, f')$ if the following properties hold:
\begin{itemize}
  \item $E_i \subset E'_i$ for $i=0,1,2$.
  \item $f$ agrees with $f'$ on $E_0$.
\end{itemize}
When this occurs we say that the partial solution $(E'_0, E'_1, E'_2, f')$ \emph{extends} the partial solution $(E_0, E_1, E_2, f)$.

We define the \emph{empty partial solution} $(E_0,E_1,E_2,f)$ by setting $E_0=E_1=E_2$ to be the empty set, and $f$ to be the empty function; it is the minimal element of the above partial order.
\end{definition}


We have the following iterative construction, that lets us add arbitrary elements to the core domain $E_0$:

\begin{lemma}[Enlarging a partial solution]\label{iteration}\uses{partial-solution} Let $(E_0, E_1, E_2, f)$ be a partial solution to \eqref{fh}, and let $h$ be an element of $\Z$ that does not lie in $E_0$.  Then there exists a partial solution $(E'_0, E'_1, E'_2, f')$ to \eqref{fh} that extends $(E_0, E_1, E_2, f)$, such that $h \in E_0'$.
\end{lemma}

\begin{proof}  Because $f$ maps $E_1 \backslash E_0$ bijectively to $E_2 \backslash E_1$, there are three cases:
\begin{itemize}
  \item $h$ is equal to an element $h_0$ of $G \backslash E_2$.
  \item $h$ is equal to an element $h_0$ of $E_1 \backslash E_0$.
  \item $h$ is equal to $h_1 = f(h_0)$ for some $h_0 \in E_1 \backslash E_0$, so that $h_1 \in E_2 \backslash E_1$.
\end{itemize}

We deal with these three cases in turn.

First suppose that $h = h_0\in G \backslash E_2$.  We perform the following construction.

\begin{itemize}
  \item Choose an element $h_1 \in \Z$ that does not lie in $E_2 \cup \{h_0\}$; this is possible because $E_2$ is finite.
  \item Choose an element $h_2 \in \Z$ such that $h_2, h_0+h_1+h_2$, and $-h_1-h_2$ are all distinct from each other and lie outside of $E_2 \cup \{h_0, h_1\}$; this is possible because $E_2$ is finite.
  \item Promote $h_0$ to $E_0$, promote $h_1, h_0+h_1+h_2$ to $E_1$, and promote $h_2, -h_1-h_2$ to $E_2$, creating new sets
  \begin{align*}
    E'_0 &:= E_0 \cup \{h_0\} \\
    E'_1 &:= E_1 \cup \{h_0, h_1, h_0+h_1+h_2\} \\
    E'_2 &:= E_2 \cup \{h_0, h_1, h_0+h_1+h_2, h_2, -h_1-h_2\}.
  \end{align*}
  Clearly we still have nested finite sets $E'_0 \subset E'_1 \subset E'_2$.
  \item Extend $f : E_1 \to E_0$ to a function $f': E'_1 \to E'_0$ by defining
  \begin{align*}
    f'(h_0) &:= h_1 \\
    f'(h_1) &:= h_2 \\
    f'(h_0+h_1+h_2) &:= -h_1-h_2
  \end{align*}
  while keeping $f'(h)=f(h)$ for all $h \in E_1$.
\end{itemize}
It is then a routine matter to verify that $(E'_0,E'_1,E'_2,f')$ is a partial solution to \eqref{fh} extending $(E_0,E_1,E_2,f)$ and that $E'_0$ contains $h_0$, as required.

Now suppose that $h = h_0 \in E_1 \backslash E_0$, then the quantity $h_1 := f(h_0)$ lies in $E_2 \backslash E_1$.  We perform the following variant of the above construction:
\begin{itemize}
\item Choose an element $h_2 \in \Z$ such that $h_2, h_0+h_1+h_2$, and $-h_1-h_2$ are all distinct and lie outside of $E_2$.  This is possible because $E_2$ is finite.
\item Promote $h_0$ to $E_0$, promote $h_1$ and $h_0+h_1+h_2$ to $E_1$, and promote $h_2, -h_1-h_2$ to $E_2$, thus creating new sets
\begin{align*}
  E'_0 &:= E_0 \cup \{h_0\} \\
  E'_1 &:= E_1 \cup \{h_1, h_0+h_1+h_2\} \\
  E'_2 &:= E_2 \cup \{h_0+h_1+h_2, h_2, -h_1-h_2\}.
\end{align*}
Clearly we still have nested finite sets $E'_0 \subset E'_1 \subset E'_2$.
\item Extend $f : E_1 \to E_0$ to a function $f': E'_1 \to E'_0$ by defining
\begin{align*}
  f'(h_1) &:= h_2 \\
  f'(h_0+h_1+h_2) &:= -h_1-h_2
\end{align*}
while keeping $f'(h)=f(h)$ for all $h \in E_1$.
\end{itemize}
It is then a routine matter to verify that $(E'_0,E'_1,E'_2,f')$ is a partial solution to \eqref{fh} extending $(E_0,E_1,E_2,f)$ and that $E'_0$ contains $h_0$, as required.

Finally, suppose that $h = h_1 = f(h_0)$ for some $h_0 \in E_1 \backslash E_0$, so that $h_1 \in E_2 \backslash E_1$.  Then we perform the following algorithm.
\begin{itemize}
  \item Choose an element $h_2 \in \Z$ such that $h_2, h_0+h_1+h_2$, and $-h_1-h_2$ are all distinct and lie outside of $E_2$.  This is possible because $E_2$ is finite.
  \item Choose an element $h_3 \in \Z$ such that $h_3, h_1+h_2+h_3$, and $-h_2-h_3$ are all distinct and lie outside of $E_2 \cup \{h_2, h_0+h_1+h_2,-h_1-h_2\}$.  This is possible because $E_2$ is finite.
  \item Promote $h_0, h_1$ to $E_0$, promote $h_2, h_0+h_1+h_2, h_1+h_2+h_3$ to $E_1$, and promote $h_3, -h_1-h_2,-h_2-h_3$ to $E_2$, creating new sets
\begin{align*}
  E'_0 &:= E_0 \cup \{h_0, h_1 \}\\
  E'_1 &:= E_1 \cup \{h_1, h_2, h_0+h_1+h_2, h_1+h_2+h_3 \}\\
  E'_2 &:= E_2 \cup \{h_2, h_3, h_0+h_1+h_2, h_1+h_2+h_3, -h_1-h_2, -h_2-h_3\}.
\end{align*}
Clearly we still have nested finite sets $E'_0 \subset E'_1 \subset E'_2$.
\item Extend $f : E_1 \to E_0$ to a function $f': E'_1 \to E'_0$ by defining
\begin{align*}
  f'(h_1) &:= h_2 \\
  f'(h_0+h_1+h_2) &:= -h_1-h_2
  f'(h_2) &:= h_3 \\
  f'(h_1+h_2+h_3) &:= -h_2-h_3
\end{align*}
while keeping $f'(h)=f(h)$ for all $h \in E_1$.
\end{itemize}
It is then a routine matter to verify that $(E'_0,E'_1,E'_2,f')$ is a partial solution to \eqref{fh} extending $(E_0,E_1,E_2,f)$ and that $E'_0$ contains $h_0$, as required.
\end{proof}

\begin{corollary} \label{extend}\uses{partial-solution} Every partial solution $(E_0,E_1,E_2,f)$ to \eqref{fh} can be extended to a full solution $\tilde f: \Z \to \Z$.
\end{corollary}

\begin{proof}\uses{iteration}
  If we arbitrarily well-order the integers, and iterate \Cref{iteration} to add the least element of $\Z \backslash E_0$ in this well-ordering to $E_0$, we obtain an increasing sequence $(E^{(n)}_0, E^{(n)}_1, E^{(n)}_2, f^{(n)})$ of partial solutions to \eqref{fh}, where the $E^{(n)}_0$ exhaust $\Z$: $\bigcup_{n=1}^\infty E^{(n)}_0 = \Z$.  Taking limits, we obtain a full solution $\tilde f$.
\end{proof}

\begin{corollary}\label{no-inject}  There exists a solution $f:\Z \to \Z$ to \eqref{fh} such that the map $h \mapsto h + f(h)$ is not injective.
\end{corollary}

\begin{proof}  Select integers $h_0,h_1,h_2,h'_0,h'_1,h'_2$ such that the quantities
  $$ h_0, h_1, h_2, h_0+h_1+h_2, -h_1-h_2, h'_0, h'_1, h'_2, h'_0+h'_1+h'_2, -h'_1-h'_2$$
  are all distinct, but such that
  $$ h_0 + h_1 = h'_0 + h'_1$$
  (there are many assignments of variables that accomplish this).  Then set
\begin{align*}
  E_0 &:= \{h_0, h'_0\}\\
  E_1 &:= E_0 \cup \{ h_1, h'_1, h_0+h_1+h_2, h'_0+h'_1+h'_2\}\\
  E_2 &:= E_2 \cup \{ -h_1-h_2, -h'_1-h'_2\}
\end{align*}
and define $f: E_1 \to E_2$ by the formulae
\begin{align*}
  f(h_0) &:= h_1 \\
  f(h_1) &:= h_2 \\
  f(h_0+h_1+h_2) &:= -h_1-h_2 \\
  f(h'_0) &:= h'_1 \\
  f(h'_1) &:= h'_2 \\
  f(h'_0+h'_1+h'_2) &:= -h'_1-h'_2.
\end{align*}
One can then check that $(E_0, E_1, E_2, f)$ is a partial solution to \eqref{fh}, and by construction $h \mapsto h + f(h)$ is not injective on $E_1$.  Using \Cref{iteration} to extend this partial solution to a full solution, we obtain the claim.
\end{proof}

\begin{corollary}  There exists a magma obeying the Asterix law (\Cref{eq65}) with carrier $\Z$ such that the left-multiplication maps $L_y: x \mapsto y \op x$ are not injective for any $y \in \Z$.  In particular, it does not obey the Obelix law (\Cref{eq1491})
\end{corollary}

\begin{proof} Note that $L_y (y+h) = y + h + f(h)$, so the injectivity of the left-multiplication maps is equivalent to the injectivity of the map $h \mapsto h + f(h)$.  The non-injectivity then follows from \Cref{no-inject}. Note that the Obelix law clearly expresses $x$ as a function of $y$ and $L_y x = y \op x$, forcing injectivity of left-multiplication, so the Obelix law fails.
\end{proof}
